\chapter{Wprowadzenie}

%---------------------------------------------------------------------------

\section{Motywacja}

Podczas tworzenia aplikacji asychronicznych twórcy wielokrotnie napotykają ograniczenia narzędzi, które nie są przystosowane do pracy z tą klasą aplikacji. Podstawowe techniki, takie jak analiza wyjątków czy klasyczne debuggery, przeważnie nie dają wystarczających informacji o naturze problemów. Iulian Dragos w swojej prezentacji \cite{rethingningDebugger} przedstawił koncepję asychronicznego debuggera, przeznaczonego do debugowania aplikacji stworzonych przy wykorzystaniu technologii Akka oraz mechanizmu Feature'ów z języka Scala. Po wysłuchaniu prelekcji uznałem, że przedstawiona przez Dragos'a koncepcja jest dobra, jednakże wykorzystane tutaj sposoby persystencji komunikatów będą miały zbyt duży wpływ na działania debuggowanej aplikacji.



\section{Cele pracy}


Celem poniższej pracy jest zbadanie możliwości oraz efektywaności debugowania aplikacji komunikujących się asynchronicznie w oparciu o historię komunikatów. Głównym obszarem zainteresowań pracy będą pomiar oraz zmieniejszenie narzutu sposobu persystowania i analizy komunikatów na czas wykonywania poszczególnych części aplikacji. Zamierzam zaimplementować i przetestować różne podejścia, a następnie zestawić wyniki z ograniczeniami danej metody. Ponieważ jednak przedmiotem tej pracy nie jest stworzenie asynchronicznego debuggera, wykorzystam pracę Iuliana \cite{asychDebuggerGh}. Zaimpelementowany debugger jest częścią ScalaIDE – IDE dedykowanego Scali. Przetestuję wydajność, wykorzystując aplikacje napisane w frameworku Akka – najpopularniejszej technologii do pisania aplikacji asychronicznych, opartych o wymianę komunikatów w ekosystemie Scali.














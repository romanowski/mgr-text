\chapter{Wprowadzenie}

%---------------------------------------------------------------------------

\section{Motywacja}

Podczas tworzenia aplikacji asychronicznych twórcy wielokrotnie napotykają ograniczenia narzędzi które nie są przystosowane do pracy z tą klasą aplikacji. Podstawowe techniki takie jak analiza wyjątków czy klasyczne debuggery przeważnie nie daje nam wystarczających informacji o naturze problemów. Iulian Dragos w swojej prezentacji  \cite{rethingningDebugger} przedstawił koncepje asychronicznego debuggera przeznaczonego do debugowania aplikacji stworzych przy wykorzystaniu technologi Akka oraz mechanizmu Feature'ów z języka Scala. Po jej wysłuchaniu uznałem że przedstawiona koncepcja jest dobra, jednakże wykorzystane sposoby persystencji komunikatów będą miały zbyt duży wpływ na działania debuggowanej aplikacji.



\section{Cele pracy}

Celem poniższej pracy jest zbadanie możliwości oraz efektywaności debuggowania aplikacji komunikujących się asychronicznie w oparciu o historię komunikatów. Głównym obszarem zainteresowań pracy będzie narzut sposobu persystowania i analizy komunikatów na czas wykonywania poszczególnych części aplikacji. Zamierzam zaimplementować, przetestować różne podejścia oraz zestawić wyniki wraz z ograniczeniami danej metody. Jako że przedmiotem tej pracy nie jest stworzenie asychronicznego debuggera zamierzam wykorzystać pracę Iuliana \cite{asychDebuggerGh}. Zaimpelemtowany debugger jest częścią ScalaIDE - IDE dedykowanego Scali. Zamierzam testować wydajność wykorzystując aplikacje napisane w frameworku Akka - najpopularniejszej technologi do pisania aplikacji asychronicznych opartych o wymiane komunikatów w ekosystemie Scali.














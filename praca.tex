\documentclass[11pt]{aghdpl}
% \documentclass[en,11pt]{aghdpl}  % praca w języku angielskim
\usepackage[polish]{babel}
%\usepackage[english]{babel}
\usepackage[utf8]{inputenc}

% dodatkowe pakiety
\usepackage{enumerate}
\usepackage{listings}
\lstloadlanguages{TeX}

\lstset{
  literate={ą}{{\k{a}}}1
           {ć}{{\'c}}1
           {ę}{{\k{e}}}1
           {ó}{{\'o}}1
           {ń}{{\'n}}1
           {ł}{{\l{}}}1
           {ś}{{\'s}}1
           {ź}{{\'z}}1
           {ż}{{\.z}}1
           {Ą}{{\k{A}}}1
           {Ć}{{\'C}}1
           {Ę}{{\k{E}}}1
           {Ó}{{\'O}}1
           {Ń}{{\'N}}1
           {Ł}{{\L{}}}1
           {Ś}{{\'S}}1
           {Ź}{{\'Z}}1
           {Ż}{{\.Z}}1
}

%---------------------------------------------------------------------------

\author{Krzysztof Romanowski}
\shortauthor{K. Romanowski}

\titlePL{Debuggowanie aplikacji kominukujących się asynchronicznie oparte o historię komunikatów.}
\titleEN{History-based approach for debugging applications using asynchronous communication.}

\shorttitlePL{Debugowanie} % skrócona wersja tytułu jeśli jest bardzo długi
\shorttitleEN{Debugging}

\thesistype{Praca dyplomowa magisterska}
%\thesistype{Master of Science Thesis}

\supervisor{dr hab. Arkadiusz Janik, dr inż}
%\supervisor{Marcin Szpyrka PhD, DSc}

\degreeprogramme{Informatyka}
%\degreeprogramme{Computer Science}

\date{2015}

\department{Katedra Informatyki}

\faculty{Wydział Elektrotechniki, Automatyki, Elektroniki i Telekomunikacji}

\acknowledgements{Serdecznie dziękuję \dots tu ciąg dalszych podziękowań np. dla promotora, żony, sąsiada itp.}


\setlength{\cftsecnumwidth}{10mm}

%---------------------------------------------------------------------------
\setcounter{secnumdepth}{4}

\begin{document}

\titlepages
\setcounter{tocdepth}{3}
\tableofcontents
\clearpage

\chapter{Wprowadzenie}

%---------------------------------------------------------------------------

\section{Motywacja}

Podczas tworzenia aplikacji asychronicznych twórcy wielokrotnie napotykają ograniczenia narzędzi, które nie są przystosowane do pracy z tą klasą aplikacji. Podstawowe techniki, takie jak analiza wyjątków czy klasyczne debuggery, przeważnie nie daja nam wystarczających informacji o naturze problemów. Iulian Dragos w swojej prezentacji  \cite{rethingningDebugger} przedstawił koncepje asychronicznego debuggera przeznaczonego do debugowania aplikacji stworzonych przy wykorzystaniu technologii Akka oraz mechanizmu Feature'ów z języka Scala. Po jej wysłuchaniu uznałem, że przedstawiona koncepcja jest dobra, jednakże wykorzystane sposoby persystencji komunikatów będą miały zbyt duży wpływ na działania debuggowanej aplikacji.



\section{Cele pracy}

Celem poniższej pracy jest zbadanie możliwości oraz efektywaności debuggowania aplikacji komunikujących się asynchronicznie w oparciu o historię komunikatów. Głównym obszarem zainteresowań pracy będzie narzucenie sposobu persystowania i analizy komunikatów na czas wykonywania poszczególnych części aplikacji. Zamierzam zaimplementować, przetestować różne podejścia oraz zestawić wyniki z ograniczeniami danej metody. Jako że przedmiotem tej pracy nie jest stworzenie asynchronicznego debuggera zamierzam wykorzystać pracę Iuliana \cite{asychDebuggerGh}. Zaimpelemtowany debugger jest częścią ScalaIDE – IDE dedykowanego Scali. Zamierzam testować wydajność wykorzystując aplikacje napisane w frameworku Akka – najpopularniejszej technologi do pisania aplikacji asychronicznych opartych o wymiane komunikatów w ekosystemie Scali.














\chapter{Asychroniczny debugger}

W tym rozdziale zamierzam przedstawić zasade działania wykorzystanego asychronicznego debuggera. Zamierzam nakreślić problemy oraz sposoby ich rozwiązywania oraz pokazać dlaczego sposób perysytencji komunikatów jest kluczowy dla minimalizacji wpływu debuggera na debuggowaną .

%---------------------------------------------------------------------------

\section{Techniki i narzędzia służących do debuggowania}

W tym podroziale przedstawie po krótce techniki oraz narzędzia służące analizie oraz debuggowaniu aplikacji które udostępnia ekosystem JVM. 

\subsection{Traceing}

\subsection{Wyjątki i ich analiza}

\subsection{Instrumentacja kodu}

\subsection{Debugger}

\section{Java Platform Debugger Architecture}

W tym podrozdziale zamierzam przedstawić Java Platform Debugger Architecture która jest podstawą dla każdego debuggera dla JVM.

\subsection{Architektura}

\subsection{JVM TI: instrumentacja JVM}

\subsection{JDW: protokół transportowy}

\subsection{JDI: wysokopoziomowe API}

\subsection{JDI: implementacja w środowisku Eclipse}


\section{Zasada dzialania}

W tym podrozdziale zamierzam przedstawić zasadedziałania asychronicznego debuggera oraz nakreślić  miejsca które będą przedmiotem tej pracy.

\subsection{Asychrnoniczna historia wywołań}
TODO
\subsection{Historia komunikatów}
TODO
\subsection{Budowanie historii komunikatów}

TODO

\subsection{Asychroniczne debuggowanie aplikacji aktorowych: Akka}
\chapter{Aplikacje testowe}

W tym rozdiale zamierzam przedstawić aplikacje które posłużą do testowania debuggera. Opiszę technologię oraz algorytmy w nich zastosowane. 

\section{Opis technologii}

\subsection{Model aktora}

\subsection{Framework Akka}

\subsection{Integracja z Asychronicznym debuggerem}

\section{Aplikacja 1}

\subsection{Motywacja}

Co chcemy zbadać, po co itp.

\subsection{Opis diałania}

Opis algorytmu, architektury itp.

\subsection{Opis debuggowania}

W jaki sposób aplikacja bedzie debuggowana, opis breakpointów, testów wydajności oraz walidacji wyników.

\section{Aplikacja 2}

\subsection{Motywacja}

Co chcemy zbadać, po co itp.

\subsection{Opis diałania}

Opis algorytmu, architektury itp.

\subsection{Opis debuggowania}

W jaki sposób aplikacja bedzie debuggowana, opis breakpointów, testów wydajności oraz walidacji wyników.
\chapter{Sposoby tworzenie historii komunikatów}

W tym rozdiale zamierzem przedstawić sposoby tworzenia historii komunikatów. Zamierzam podzielić ten proces na dwa etapy i przedstawić sposoby ich implementacji. W ostatnim podrozdiale zamierzam przedstawić testowane sposoby (złożenia). 

\section{Dwa etapy: zbieranie i wysyłanie danych}

\section{Metody zbierania danych}

\subsection{Plain JDI}

\subsection{JVM TI: Java Agent}

\subsection{JDI: instrumentacja kodu}

\section{Metody przesyłania danych}

\subsection{Plain JDI}

\subsection{Plain JDI: lazy mode}

\subsection{Filesystem}

\subsection{Socets}

\section{Testowane złożenia}

TODO: wyjdzie podczas implementacji


\chapter{Wyniki oraz ich a analiza}

\section{Aplickacja 1}
\subsection{Test 1}
\subsection{Test 2}
\subsection{Test 3}
\subsection{Analiza}

\section{Aplickacja 2}
\subsection{Test 1}
\subsection{Test 2}
\subsection{Test 3}
\subsection{Analiza}

\section{Zestawienie zbiorcze}
\subsection{Test 1}
\subsection{Test 2}
\subsection{Test 3}
\subsection{Analiza}



\chapter{Analiza oraz wnioski}

TODO: w zależności co wyjdzie

\section{Dalsze możlowości rozwoju}




% itd.
% \appendix
% \include{dodatekA}
% \include{dodatekB}
% itd.

\bibliographystyle{alpha}
\bibliography{bibliografia}
%\begin{thebibliography}{1}
%
%\bibitem{Dil00}
%A.~Diller.
%\newblock {\em LaTeX wiersz po wierszu}.
%\newblock Wydawnictwo Helion, Gliwice, 2000.
%
%\bibitem{Lam92}
%L.~Lamport.
%\newblock {\em LaTeX system przygotowywania dokumentów}.
%\newblock Wydawnictwo Ariel, Krakow, 1992.
%
%\bibitem{Alvis2011}
%M.~Szpyrka.
%\newblock {\em {On Line Alvis Manual}}.
%\newblock AGH University of Science and Technology, 2011.cccccc
%\newblock \\\texttt{http://fm.ia.agh.edu.pl/alvis:manual}.
%
%\end{thebibliography}

\end{document}

\chapter{Asychroniczny debugger}

W tym rozdziale zamierzam przedstawić zasade działania wykorzystanego asychronicznego debuggera. Zamierzam nakreślić problemy oraz sposoby ich rozwiązywania oraz pokazać dlaczego sposób perysytencji komunikatów jest kluczowy dla minimalizacji wpływu debuggera na debuggowaną .

%---------------------------------------------------------------------------

\section{Techniki i narzędzia służących do debuggowania}

W tym podroziale przedstawie po krótce techniki oraz narzędzia służące analizie oraz debuggowaniu aplikacji które udostępnia ekosystem JVM. 

\subsection{Traceing}

\subsection{Wyjątki i ich analiza}

\subsection{Instrumentacja kodu}

\subsection{Debugger}

\section{Java Platform Debugger Architecture}

W tym podrozdziale zamierzam przedstawić Java Platform Debugger Architecture która jest podstawą dla każdego debuggera dla JVM.

\subsection{Architektura}

\subsection{JVM TI: instrumentacja JVM}

\subsection{JDW: protokół transportowy}

\subsection{JDI: wysokopoziomowe API}

\subsection{JDI: implementacja w środowisku Eclipse}


\section{Zasada dzialania}

W tym podrozdziale zamierzam przedstawić zasadedziałania asychronicznego debuggera oraz nakreślić  miejsca które będą przedmiotem tej pracy.

\subsection{Asychrnoniczna historia wywołań}
TODO
\subsection{Historia komunikatów}
TODO
\subsection{Budowanie historii komunikatów}

TODO

\subsection{Asychroniczne debuggowanie aplikacji aktorowych: Akka}